\section{Conclusiones}

Se ha llegado a la conclusión final de que las funciones biológicas de las proteínas estudiadas están basadas en el daño que el SarsCOV-2 le genera a los pulmones. 
Como se ha explicado, la fucosa los inflama y les genera un daño permanente que hace que las mitocondrias de las células se dañen y provoquen la muerte celular, lo que impide la regeneración de las matrices extracelulares de los pulmones.
Esto desencadenará en daños permanentes en los pacientes, lo que puede impedir su total recuperación, que se les desarrollen patologías crónicas o incluso provocarles la muerte (sepsis por COVID).

Los resultados obtenidos en el análisis concuerdan con la descripción del propio virus, ya que el SarsCOV-2 provoca una infección respiratoria en los pacientes que afecta a todo su sistema respiratorio. 

Este resultado demuestra la importancia de contar con el interactoma humano a la hora de estudiar enfermedades y así encontrar los principales focos causantes de la misma.