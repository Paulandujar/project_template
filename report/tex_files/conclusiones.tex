\section{Conclusiones}

Se ha llegado a la conclusión final de que las funciones biológicas de las proteínas estudiadas están basadas en el daño que el SarsCOV-2 le genera a los pulmones. 
Como se ha explicado, la fructosa los inflama y les genera un daño permanente que hace que las mitocondrias de las células se dañen y provoquen la muerte celular, lo que impide la regeneración de las matrices extracelulares de los pulmones.
Esto desencadenará en daños permanentes en los pacientes, lo que puede impedir su total recuperación, que se les desarrollen patologías crónicas o incluso que le provoquen la muerte (sepsis por COVID).