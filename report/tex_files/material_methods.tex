\section{Materiales y métodos}
2.1 Carga de librerías y datos
Para poder realizar este trabajo, el primer paso a realizar es la carga de librerías necesarias y la carga de datos. Antes de cargar los datos, estos han sido descargados de Uniprot (https://www.uniprot.org/) en formato .csv para poder llevar a cabo el análisis de la red.
Una vez ha sido añadido este fichero al directorio correspondiente (data), se procederá a la carga de librerías.
Las librerías que han sido utilizadas en este proyecto son las siguientes:
\begin{itemsize}
	\item igraph: Esta librería permite realizar análisis de redes, por lo cual, proporciona funciones para manipular gráficos con facilidad.
	\item dplyr: Esta librería proporciona métodos para poder manejar los ficheros de datos.
	\item ggplot2: Esta librería es un paquete de visualización de datos.
	\item zoo: Esta librería está especialmente dirigida a series temporales irregulares de vectores/matrices y factores numéricos. 
	\item STRINGdb: Este paquete proporciona una interfaz para la base de datos STRING de interacciones proteína-proteína.
\end{itemsize}
	



2.2 Análisis inicial y robustez

2.3 Linked Communities

2.4 Enriquecimiento funcional

