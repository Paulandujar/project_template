\section{Materiales y métodos}
\subsection{Carga de librerías y datos}
Para poder llevar a cabo este trabajo, el primer paso a realizar es la carga de librerías necesarias y la carga de datos. Antes de cargar los datos, estos han sido descargados de Uniprot (https://www.uniprot.org/) en formato .csv para poder llevar a cabo el análisis de la red.
Una vez ha sido añadido este fichero al directorio correspondiente (data), se procederá a la carga de librerías.
Las librerías que han sido utilizadas en este proyecto son las siguientes:
\begin{itemize}
	\item igraph: Esta librería permite realizar análisis de redes, por lo cual, proporciona funciones para manipular gráficos con facilidad.
	\item dplyr: Esta librería proporciona métodos para poder manejar los ficheros de datos.
	\item ggplot2: Esta librería es un paquete de visualización de datos.
	\item zoo: Esta librería está especialmente dirigida a series temporales irregulares de vectores/matrices y factores numéricos. 
	\item STRINGdb: Este paquete proporciona una interfaz para la base de datos STRING de interacciones proteína-proteína.
\end{itemize}

Después de tener las librerías necesarias y saber la funcionalidad de cada una de ellas, se procederá a cargar el archivo en una variable llamada "data" mediante el método "read.csv()"

Seguidamente, se filtrarán las entradas utilizando el paquete "dplyr" en las que la columna Entry.Name contenga en su nombre "HUMAN" ya que estos son los datos que interesan en esta práctica.

\subsection{Mapeo y primera capa de la red}
A continuación, utilizando la librería STRINGdb, se realiza un mapeo con los datos ya filtrados. Se guardará los hits de string en una imagen png en el directorio de los resultados (results)

Seguidamente, se creará la primera capa de la red y se guardará el resultado de esta primera capa en una imagen png. 

\subsection{Clustering y Linked Communities}
Una vez se tiene los datos ya mapeados y filtrados, se pasa a realizar el clustering. El clustering consiste en agrupar los ítemns en grupos con características similares y así, determinar patrones.
Utilizando el paquete STRINGdb, los datos mapeados han sido clasificados en cuatro grupos diferentes según sus características.

Seguidamente, utilizando el paquete linkcommm, se ha realizado la búsqueda de comunidades. Linkcomm proporciona las herramientas necesarias para generar, visualizar y analizar comunidades dentro de un grafo.

Al obtener las comunidades vinculadas, estas serán guardadas en la carpeta de results y, además, se han obtenido los tamaños de los clusters y han sido clasificados por comunidad/modularidad para así obtener los 6 mejores.

\subsection{Enriquecimiento Funcional}
Al realizar el clustering y el agrupamiento de estas comunidades, se realizará el enriquecimiento funcional. Este enriquecimiento es utilizado para obtener los procesos biológicos de los grupos que han sido formados en el clustering.
En primer lugar, ha sido diseñado el enriquecimiento mediante STRINGdb. Después, se ha realizado el enriquecimiento con GO, es decir, con una ontología génica y también, se ha realizado el enriquecimiento con la ontología KEGG. 

El enriquecimiento ha sido utilizado con los clusters de mayor tamaño y de mayor modularidad.

\subsection{Robustez}
Por último, utilizando los métodos proporcionados en el campus virtual, se ha calculado la robustez de la red de genes. Esta funciona para conocer si la red que se está estudiando es un sistema fuerte y esta no tiende a tener errores. Además, ha sido calculada frente a ataques aleatorios como a ataques dirigidos, pero también, han sido combinados ambos ataques.

 