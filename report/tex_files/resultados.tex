\section{Resultados}

\subsection{Red de interacciones y robustez}

La siguiente imagen muestra la red de interaciones del ser humano con las proteínas del SARS-CoV. Como podemos ver el SARS-CoV interacciona con 89 proteínas humanas, produciendo un total de 475 interacciones. 
\begin{figure}
	\centering
	
	\includegraphics[width=70mm,scale=1.2]{figures/string_hits.png}
	
	\caption{\textit{Red de interacciones del SARS-CoV con las proteínas humanas}}
	
\end{figure}

Tras eliminar los nodos que no están conectados, hemos obtenido la red real de interacciones que podemos ver a continuación. Sin embargo hay demasiadas conexiones como para poder distinguir los nodos. Es por ello que realizaremos los pasos siguientes de clustering, para así poder extraer la información relevante de la red. 
\begin{figure}
	\centering
	
	\includegraphics[width=70mm,scale=1.2]{figures/hits.network_graph.png}
	
	\caption{\textit{Red de interacciones del SARS-CoV con las proteínas humanas tras un proceso de filtrado}}
	
\end{figure}


Antes de empezar con ese proceso vamos a estudiar diferentes aspectos de nuestra red. En primer lugar si observamos la imagen, vemos que el la distribución de grado sigue la ley de potencias, por lo tanto nuestra red sigue un modelo de free-scale, lo cual era predecible al estar tratando con una red real. 

Podemos  ver una gran cantidad de hubs. 
\begin{figure}
	\centering
	
	\includegraphics[width=70mm,scale=1.2]{figures/degree_distribution.png}
	
	\caption{\textit{Distribución de grado}}
	
\end{figure}

El coeficiente medio de agrupamiento es de 0.605, lo cual es bastante alto. Además se puede observar la característica propia de las redes reales la cual afirma que conforme el grado de los nodos aumenta, el coeficiente de agrupamiento disminuye. 

\begin{figure}
	\centering
	
	\includegraphics[width=70mm,scale=1.2]{figures/coeficiente_agrupamiento.png}
	
	\caption{\textit{Coeficiente de Agrupamiento}}
	
\end{figure}


La distancia media entre nodos es de 2.03, una medida muy pequeña que puede significar que los nodos tienen un alto índice de conexiones. Esta medida es la que le da la propiedad de mundo pequeño, es decir, la distancia entre nodos elegidos al azar en una red es muy pequeña. 
\begin{figure}
	\centering
	
	\includegraphics[width=70mm,scale=1.2]{figures/distancia.png}
	
	\caption{\textit{Distancia entre nodos}}
	
\end{figure}

Por último vamos a estudiar la robustez de nuestra red. Para poder estudiar cual es la capacidad de nuestra red de mantener sus funciones frente a la presencia de "ataques" y ver cuán de adaptable es, usamos la robustez. Podemos observar que para ataques aleatorios es bastante robusta, mientras que para ataques dirigidos es más débil. Pues a que tenemos una red real, estos resultados resultan obvios, ya que es más fácil destruir una red si atacas a puntos estratégicos como son los hubs, dónde el tamaño de la componente conexa se reduce drásticamente cuando eliminamos una pequeña fracción de los nodos (hubs). 

\begin{figure}
	\centering
	\includegraphics[width=70mm,scale=1.2]{figures/sequential_attacks.png}
	\caption{\textit{Robustez frente a ataques dirigidos y aleatorios}}
\end{figure}

\subsection{Enriquecimiento funcional}

En esta sección se van a mostrar los resultados obtenidos al realizar el enriquecimiento funcional con GO y con KEGG, mediante el uso de STRINGdb, para los clústeres elegidos.
Se ha guardado la información del enriquecimiento en archivos de tipo csv, y se va a mostrar una imagen de los mismos.

\subsubsection{Clúster 112}

\paragraph{Enriquecimiento con GO}

\begin{figure}
	\centering
	\includegraphics[width=70mm,scale=1.2]{figures/cluster112_GO.png}
	\caption{\textit{Funciones biológicas clúster 112 con GO}}
\end{figure}

Se puede observar que las funciones biológicas visibles están relacionadas con la defensa del organismo, por lo que se puede deducir que son proteínas que forman parte del sistema inmunológico de nuestro organismo. No obstante, en la imagen solo aparecen algunas funciones biológicas. Si se sigue observando el archivo generado nos encontramos con:

\begin{figure}
	\centering
	\includegraphics[width=70mm,scale=1.2]{figures/cluster112_GO_2.png}
	\caption{\textit{Funciones biológicas clúster 112 con GO}}
\end{figure}

Aquí observamos que otras tantas funciones biológicas están asociadas con la regulación de diversos procesos biológicos.

\paragraph{Enriquecimiento con KEGG}

\begin{figure}
	\centering
	\includegraphics[width=70mm,scale=1.2]{figures/cluster112_KEGG.png}
	\caption{\textit{Funciones biológicas clúster 112 con KEGG}}
\end{figure}

Se puede observar que las funciones biológicas están relacionadas con enfermedades/infecciones (malaria, hepatitis B, tuberculosis) por lo que se podríamos deducir que dichas proteínas forman parte de la respuesta inmunitaria del organismo ante dichas enfermedades o que las provocan.
Además de esto, se observan funciones biológicas relacionadas con vías biológicas del organismo, como son la vía de señalización de quimioquinas, la de detección de ADN ctosólico...

\subsubsection{Clúster 78}

\paragraph{Enriquecimiento con GO}

\begin{figure}
	\centering
	\includegraphics[width=70mm,scale=1.2]{figures/cluster78_GO.png}
	\caption{\textit{Funciones biológicas clúster 78 con GO}}
\end{figure}

En este resultado, GO no ha encontrado funciones biológicas asociadas al clúster elegido.

\paragraph{Enriquecimiento con KEGG}

\begin{figure}
	\centering
	\includegraphics[width=70mm,scale=1.2]{figures/cluster78_KEGG.png}
	\caption{\textit{Funciones biológicas clúster 78 con KEGG}}
\end{figure}

En este caso, KEGG ha encontrado dos funciones biológicas asociadas al clúster: cáncer de tiroides y vías en el cáncer. Es por tanto que deducimos que las proteínas que forman parte de este clúster se ocupan de las vías principales del cáncer de tiroides, ya sea para detectarlo o provocarlo.

\subsubsection{Clúster 22}

\paragraph{Enriquecimiento con GO}

\begin{figure}
	\centering
	\includegraphics[width=70mm,scale=1.2]{figures/cluster22_GO.png}
	\caption{\textit{Funciones biológicas clúster 22 con GO}}
\end{figure}

En este caso, se puede observar que las principales funciones biológicas de las proteínas pertenecientes a este clúster son de regulación, transporte y recepción, por lo que desempeñan funciones muy importantes en las rutas metabólicas del organismo. 

\paragraph{Enriquecimiento con KEGG}

\begin{figure}
	\centering
	\includegraphics[width=70mm,scale=1.2]{figures/cluster22_KEGG.png}
	\caption{\textit{Funciones biológicas clúster 22 con KEGG}}
\end{figure}

Las funciones biológicas obtenidas por KEGG son más concretas: reabsorción de calcio regulada por factores endocrinos, ciclo de vesículas sinápticas, endocitosis, lisosoma, enfermedad de Huntington, invasión bacteriana de las células epiteliales y vía de señalización de la fosfolipasa D. Esta última está relacionada con la traducción de señales. Otras están relacionadas con enfermedades y sus causas.

\subsection{Clustering}


