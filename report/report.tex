\documentclass{bmcart}

%%%%%%%%%%%%%%%%%%%%%%%%%%%%%%%%%%%%%%%%%%%%%%
%%                                          %%
%% CARGA DE PAQUETES DE LATEX               %%
%%                                          %%
%%%%%%%%%%%%%%%%%%%%%%%%%%%%%%%%%%%%%%%%%%%%%%

%%% Load packages
\usepackage{amsthm,amsmath}
\usepackage{graphicx}
%\RequirePackage[numbers]{natbib}
%\RequirePackage{hyperref}
\usepackage[utf8]{inputenc} %unicode support
\usepackage{listings}
%\usepackage[applemac]{inputenc} %applemac support if unicode package fails
%\usepackage[latin1]{inputenc} %UNIX support if unicode package fails


%%%%%%%%%%%%%%%%%%%%%%%%%%%%%%%%%%%%%%%%%%%%%%
%%                                          %%
%% COMIENZO DEL DOCUMENTO                   %%
%%                                          %%
%%%%%%%%%%%%%%%%%%%%%%%%%%%%%%%%%%%%%%%%%%%%%%

\begin{document}

	\begin{frontmatter}
	
		\begin{fmbox}
			\dochead{Research}
						
			
			\title{Red de los targets de SARS-CoV2}
			
			%%%%%%%%%%%%%%%%%%%%%%%%%%%%%%%%%%%%%%%%%%%%%%
			%% AUTORES. METER UNA ENTRADA AUTHOR        %%
			%% POR PERSONA                              %%
			%%%%%%%%%%%%%%%%%%%%%%%%%%%%%%%%%%%%%%%%%%%%%%
			
			\author[
			  addressref={aff1},
			  corref={aff1},
			  email={ireero99@uma.es}
			]{\inits{I.R.G}\fnm{Irene} \snm{Romero Granados}}
			\author[
			  addressref={aff1},
			  corref={aff1},
			  email={paulandujar@uma.es}
			]{\inits{P.A.Z}\fnm{Paula} \snm{Andújar Zambrano}}
			\author[
			addressref={aff1},
			corref={aff1},
			email={0619884107@uma.es}
			]{\inits{R.G.M}\fnm{Rosario} \snm{García Morales}}
			\author[
			addressref={aff1},
			corref={aff1},
			email={delcastillosoledad@uma.es}
			]{\inits{S.dC.C}\fnm{Soledad} \snm{del Castillo Carrera}}
			
			%%%%%%%%%%%%%%%%%%%%%%%%%%%%%%%%%%%%%%%%%%%%%%
			%% AFILIACION. NO TOCAR                     %%
			%%%%%%%%%%%%%%%%%%%%%%%%%%%%%%%%%%%%%%%%%%%%%%
			
			\address[id=aff1]{%                           % unique id
			  \ordiv{ETSI Informática},             % department, if any
			  \orgname{Universidad de Málaga},          % university, etc
			  \city{Málaga},                              % city
			  \cny{España}                                    % country
			}
		
		\end{fmbox}% comment this for two column layout
		
		\begin{abstractbox}
		
			\begin{abstract} % abstract
			Este proyecto pretende estudiar las interacciones que se producen entre las 29 proteínas del virus SARS-COV-2 y el interactoma humano. Para ello, se recopilarán los datos de interacción comentados, con los que se procederá a realizar un análisis sobre ellos y construir la red de proteínas que conforman los objetivos principales para el virus. En cuanto a las herramientas que se utilizarán, serán la base de datos biológica UniProt para la obtención de los datos de entrada, y el lenguaje de R para los métodos de análisis.
			
			\end{abstract}
			
			%%%%%%%%%%%%%%%%%%%%%%%%%%%%%%%%%%%%%%%%%%%%%%
			%% PALABRAS CLAVE DEL PROYECTO              %%
			%%%%%%%%%%%%%%%%%%%%%%%%%%%%%%%%%%%%%%%%%%%%%%
			
			\begin{keyword}
			\kwd{SARS-COV-2}
			\kwd{interactoma}
			\kwd{R}
			\end{keyword}
		
		
		\end{abstractbox}
	
	\end{frontmatter}
	
	
	%%%%%%%%%%%%%%%%%%%%%%%%%%%%%%%%%
	%% COMIENZO DEL DOCUMENTO REAL %%
	%%%%%%%%%%%%%%%%%%%%%%%%%%%%%%%%%
	
	\section{Introducción}
La familia de los coronavirus son virus infecciosos a los que se llama así debido a que en su superficie tienen puntas en forma de corona. A esta familia se les unió en 2019 el conocido SARS-CoV-2, que ha dado lugar al coronavirus 2 o COVID-19. Esta enfermedad es una enfermedad infecciosa que afecta a las vías respiratorias, de manera leve a moderada. Sin embargo esta enfermedad en personas mayores o con patologías previas puede hacer que se desarrolle la enfermedad con consecuencias o síntomas más graves, pudiendo producir hasta la muerte.

El coronavirus actualmente es considerado un problema de salud global, ya que debido a esta pandemia se han contagiado hasta ahora unas 369.955.862 personas y han fallecido un total de 5.650.738 personas. 

Es por esto que es esencial el estudio de este virus, tanto de sus genes, sus proteínas o como interacciona con el ser humano. 

A día de hoy tras toda la inversión mundial que se ha hecho para poder poner fin a este virus, se sabe que el SARS-CoV-2 está formado por 29 proteínas que interactúan con las células del ser humano pudiendo producir síntomas respiratorios graves hasta poder causar la muerte. A estas interacciones moleculares binarias proteína-proteína se les llama interactoma. 

El interactoma sirve como de punto de partida para estudiar los posibles fármacos que podrían bloquear dichas interacciones y así evitar que el virus entre a la célula y se replique. Gracias al estudio del interactoma ha sido posible la realización de vacunas contra el COVID-19. 

En este proyecto vamos a crear y estudiar la red de interacciones de las proteínas del SARS-CoV-2 con las proteínas humanas, y así poder obtener cuales son las principales funciones biológicas humanas en las que este virus interviene y relacionarlo con la realidad. Todos los recursos usados para la obtención de dicha información la podremos encontrar en el GitHub proporcionado. 



	\section{Materiales y métodos}
2.1 Carga de librerías y datos
Para poder realizar este trabajo, el primer paso a realizar es la carga de librerías necesarias y la carga de datos. Antes de cargar los datos, estos han sido descargados de Uniprot (https://www.uniprot.org/) en formato .csv para poder llevar a cabo el análisis de la red.
Una vez ha sido añadido este fichero al directorio correspondiente (data), se procederá a la carga de librerías.
Las librerías que han sido utilizadas en este proyecto son las siguientes:
\begin{itemsize}
	\item igraph: Esta librería permite realizar análisis de redes, por lo cual, proporciona funciones para manipular gráficos con facilidad.
	\item dplyr: Esta librería proporciona métodos para poder manejar los ficheros de datos.
	\item ggplot2: Esta librería es un paquete de visualización de datos.
	\item zoo: Esta librería está especialmente dirigida a series temporales irregulares de vectores/matrices y factores numéricos. 
	\item STRINGdb: Este paquete proporciona una interfaz para la base de datos STRING de interacciones proteína-proteína.
\end{itemsize}
	



2.2 Análisis inicial y robustez

2.3 Linked Communities

2.4 Enriquecimiento funcional


	
\section{Resultados}

\subsection{Red de interacciones y robustez}

La siguiente imagen (Imagen1) muestra la red de interaciones del ser humano con las proteínas del SARS-CoV. Como podemos ver el SARS-CoV interacciona con 89 proteínas humanas, produciendo un total de 475 interacciones. 
\begin{figure}
	\centering
	
		\includegraphics[width=70mm,scale=1.2]{figures/string_hits.png}
		
		\caption{\textit{Imagen1. Red de interacciones del SARS-CoV con las proteínas humanas}}
		
\end{figure}

Tras eliminar los nodos que no están conectados, hemos obtenido la red real de interacciones que podemos ver a continuación(imagen2). Sin embargo hay demasiadas conexiones como para poder distinguir los nodos. Es por ello que realizaremos los pasos siguientes de clustering, para así poder extraer la información relevante de la red. 
\begin{figure}
	\centering
	
		\includegraphics[width=70mm,scale=1.2]{figures/hits.network_graph.png}
		
		\caption{\textit{Imagen2. Red de interacciones del SARS-CoV con las proteínas humanas tras un proceso de filtrado}}

\end{figure}


Antes de empezar con ese proceso vamos a estudiar diferentes aspectos de nuestra red. En primer lugar si observamos la imagen(imagen3), vemos que el la distribución de grado sigue la ley de potencias, por lo tanto nuestra red sigue un modelo de free-scale, lo cual era predecible al estar tratando con una red real. 

Podemos  ver una gran cantidad de hubs. 
\begin{figure}
	\centering
	
		\includegraphics[width=70mm,scale=1.2]{figures/degree_distribution.png}
		
		\caption{\textit{Imagen3. Distribución de grado}}
		
\end{figure}

El coeficiente medio de agrupamiento es de 0.605, lo cual es bastante alto. Además se puede observar la característica propia de las redes reales la cual afirma que conforme el grado de los nodos aumenta, el coeficiente de agrupamiento disminuye. 

\begin{figure}
	\centering
	
	\includegraphics[width=70mm,scale=1.2]{figures/coeficiente_agrupamiento.png}
	
	\caption{\textit{Imagen4. Coeficiente de Agrupamiento}}
	
\end{figure}


Al observar la distancia entre nodos, 
\begin{figure}
	\centering
	
	\includegraphics[width=70mm,scale=1.2]{figures/distancia.png}
	
	\caption{\textit{Imagen4. Coeficiente de Agrupamiento}}
	
\end{figure}

Para poder estudiar cual es la capacidad de nuestra red de mantener sus funciones frente a la presencia de "ataques" y ver cuán de adaptable es, usamos la robustez. Podemos observar que para ataques aleatorios es bastante robusta, mientras que para ataques dirigidos es más débil. 

\begin{figure}
	\centering
		\includegraphics[width=70mm,scale=1.2]{figures/sequential_attacks.png}
		\caption{\textit{Robustez frente a ataques dirigidos y aleatorios}}
\end{figure}

\subsection{Clustering}





	\section{Discusión}

Una vez obtenidos los resultados, en este apartado se indagará en la relación que pueda tener el SarsCOV-2 con las funciones biológicas obtenidas previamente.

Tras el análisis funcional del \textbf{cluster 112}, descubrimos que una de las funciones con las que está muy relacionado es con el proceso metabólico de la fucosa. La fucosa es un azúcar que forma parte de algunas de las glucoproteínas que se encuentran en el aparato del Golgi cuando se produce la glucosilación.

Un estudio de Stanford Medicine (Estados Unidos) ha descubierto que aquellos pacientes que tenían una deficiencia en la fucosa, sufrían la enfermedad con más gravedad que los que tenían niveles normales. 
Además estos pacientes con leves niveles de fucosa, sus células inmunitarias presentaban niveles muy altos de unos receptores llamados CD16a, los cuales se sabe que aumentan la actividad inflamatoria de las células inmunitarias. 
Para una correcta respuesta inmunitaria es necesaria un poco de inflamación, sin embargo, si es demasiada puede producir que el paciente no tenga una respuesta inmunitaria buena y producir una  inflamación en los pulmones, lo que puede provocar que el paciente pueda llegar a un estado crítico. 

Así mismo, tras el análisis funcional del \textbf{cluster 22}, se ha llegado a la conclusión de que la principal función biológica que desempeñan las proteínas pertenecientes a dicho grupo afectan a la mitocondria de las células. El SarsCOV-2 secuestra las mitocondrias de las células inmunitarias, se replica dentro de las estructuras mitocondriales y altera la dinámica mitocondrial que conduce a la muerte celular, lo que aumenta la mortalidad de los pacientes que lo sufren.

De la misma forma, tras el análisis funcional del \textbf{cluster 78}, se ha llegado a la conclusión de que la principal función biológica que desempeñan las proteínas pertenecientes a dicho grupo afectan a la remodelación/degradación de la matriz extracelular de los pulmones. Los procesos patológicos que conducen a la disminución de la función pulmonar se usan para identificar mejor a los pacientes infectados con SARS-Co-V2 con mayor riesgo de deterioro agudo o daño fibrótico persistente del pulmón y, como consecuencia, se podría usar para guiar las decisiones de tratamiento.
	\section{Conclusiones}

Se ha llegado a la conclusión final de que las funciones biológicas de las proteínas estudiadas están basadas en el daño que el SarsCOV-2 le genera a los pulmones. 
Como se ha explicado, la fucosa los inflama y les genera un daño permanente que hace que las mitocondrias de las células se dañen y provoquen la muerte celular, lo que impide la regeneración de las matrices extracelulares de los pulmones.
Esto desencadenará en daños permanentes en los pacientes, lo que puede impedir su total recuperación, que se les desarrollen patologías crónicas o incluso provocarles la muerte (sepsis por COVID).

Los resultados obtenidos en el análisis concuerdan con la descripción del propio virus, ya que el SarsCOV-2 provoca una infección respiratoria en los pacientes que afecta a todo su sistema respiratorio. 

Este resultado demuestra la importancia de contar con el interactoma humano a la hora de estudiar enfermedades y así encontrar los principales focos causantes de la misma.
	
	
	%%%%%%%%%%%%%%%%%%%%%%%%%%%%%%%%%%%%%%%%%%%%%%
	%% OTRA INFORMACIÓN                         %%
	%%%%%%%%%%%%%%%%%%%%%%%%%%%%%%%%%%%%%%%%%%%%%%
	
	\begin{backmatter}
	
		\section*{Abreviaciones}%% if any
			
		
		\section*{Disponibilidad de datos y materiales}%% if any
			https://github.com/Paulandujar/project\_template
		
		\section*{Contribución de los autores}
			Usando las iniciales que habéis definido al comienzo del documento, debeis indicar la contribución al proyecto en el estilo:
			J.E : Encargado del análisis de coexpresión con R, escritura de resultados; J.R.S : modelado de red con python y automatizado del código, escritura de métodos; ...
			OJO: que sea realista con los registros que hay en vuestros repositorios de github. 
		
		
		%%%%%%%%%%%%%%%%%%%%%%%%%%%%%%%%%%%%%%%%%%%%%%%%%%%%%%%%%%%%%%%%%%%%%%%%%%%%%%%%%%%%%%%%
		%% BIBLIOGRAFIA: no teneis que tocar nada, solo sustituir el archivo bibliography.bib %%
		%% por el que hayais generado vosotros                                                %%
		%%%%%%%%%%%%%%%%%%%%%%%%%%%%%%%%%%%%%%%%%%%%%%%%%%%%%%%%%%%%%%%%%%%%%%%%%%%%%%%%%%%%%%%%
		
		\bibliographystyle{bmc-mathphys} % Style BST file (bmc-mathphys, vancouver, spbasic).
		\bibliography{bibliography}      % Bibliography file (usually '*.bib' )
	
	\end{backmatter}
\end{document}
